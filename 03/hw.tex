
\documentclass[13pt,a4paper]{scrartcl}
\usepackage[utf8]{inputenc}
\usepackage[english,russian]{babel}
\usepackage{indentfirst}
\usepackage{graphicx}
\usepackage{amsmath}
\usepackage{amssymb}
\usepackage{listings}
\usepackage{array}

% Некоторые множества

\def\Q{\mathbb{Q}}
\def\Z{\mathbb{Z}}
\def\N{\mathbb{N}}
\def\R{\mathbb{R}}
\def\C{\mathbb{C}}

% Бинарные операции над множествами

% xor
\def\xor{\oplus}
% объединение
\def\u{\cup}
% объединение
\def\i{\cap}


% Комбинаторика

% Биномиальный коэффициент : (из n  по k)
\def\set{\binom}
% ((из n по k))
\def\mset#1#2{\ensuremath{\left(\kern-.3em\left(\genfrac{}{}{0pt}{}{#1}{#2}\right)\kern-.3em\right)}}


% Опеации над несколькими множествами

% Сумма
\def\suml{\sum\limits}

% Сумма
\def\intl{\int\limits}

% Перемножение (знак П)
\def\prodl{\mul\limits}

% Объединение
\def\U{\bigcup}
\def\Ul#1#2#3{\U\limits_{{#1}={#2}}^{#3}}
\def\Ui#1#2{\Ul{#1}{#2}{\inf}}
\def\Uin#1#2{\U\limits_{{#1} \in {#2}}}

% Пересечение
\def\I{\bigcap}
\def\Il#1#2#3{\I\limits_{{#1}={#2}}^{#3}}
\def\Ii#1#2{\Il{#1}{#2}{\inf}}
\def\Iin#1#2{\I\limits_{{#1} \in {#2}}}


% Разделители

\def\ms{\medskip}
\def\bs{\bigskip}


% Греческий алфавит

\def\a{\alpha}
\def\b{\beta}
\def\g{\gamma}
\def\l{\lambda}
\def\e{\varepsilon}
\def\eps{\varepsilon}
\def\d{\delta}
\def\m{\mu}
\def\p{\phi}

\def\L{\Lambda}
\def\D{\Delta}
\def\M{\Mu}
\def\P{\Phi}


% Кванторы

\def\A{\forall}
\def\E{\exists\;}


% Что-то еще

\def\inf{\t{+}\infty}    % +inf
\def\O{\mathcal{O}}      %
\def\t{\text}
\def\bs{\textbackslash{}}


\begin{document}

\def\th{\theta}
\def\prodl{\prod\limits}
\def\dd#1#2{\frac{\partial{#1}}{\partial{#2}}}
\def\maxl{\max\limits}

\section*{\text{ Достаточные }\allowbreak \text{статистики }\allowbreak (29.09\text{ -- }06.10)}

\subsection*{\text{ Задача}\allowbreak }

\def\Xn{X_{(n)}}

\(\text{Пусть }\allowbreak X_1,..., X_n\text{ -- }\text{выборка }\allowbreak \text{из }\allowbreak \text{равномерного }\allowbreak \text{распределения }\allowbreak \text{на }\allowbreak \text{конечном }\allowbreak \text{множестве }\allowbreak \{ 1,..., \th \},\)
\(\text{где }\allowbreak \th\text{ -- }\text{натуральный }\allowbreak \text{параметр. }\allowbreak \text{Докажите, }\allowbreak \text{что }\allowbreak \text{статистика}\allowbreak \)

\(\frac{\Xn^{n + 1} - (\Xn - 1)^{n + 1} }{\Xn^n - (\Xn - 1)^n}\)
\medskip

\(\text{является }\allowbreak \text{эффективной }\allowbreak \text{оценкой }\allowbreak \text{параметра }\allowbreak \th\text{ в }\allowbreak \text{классе }\allowbreak \text{несмещенных }\allowbreak \text{оценок.}\allowbreak \)

\subsection*{\text{ Решение}\allowbreak }

\subsubsection*{\text{ Несмещенность}\allowbreak }

\(\text{Сначала }\allowbreak \text{докажем, }\allowbreak \text{что }\allowbreak \text{данная }\allowbreak \text{оценка }\allowbreak \text{является }\allowbreak \text{несмещенной.}\allowbreak \)

\(P[\Xn = k]\)
\(= P[\Xn \le  k] - P[\Xn \le  k - 1]\)
\(= (\frac{k }{\th})^n - (\frac{k - 1 }{\th})^n\)

\(E \frac{\Xn^{n + 1} - (\Xn - 1)^{n + 1} }{\Xn^n - (\Xn - 1)^n}\)
\(= \suml_{k = 1}^{\th} \frac{k^{n + 1} - (k - 1)^{n + 1} }{k^n - (k - 1)^n}\cdot  P[\Xn = k]\)
\(= \suml_{k = 1}^{\th} \frac{k^{n + 1} - (k - 1)^{n + 1} }{\th^n}\)
\(= \frac{1 }{\th^n}\cdot  (\suml_{k = 1}^{\th} k^{n + 1} - \suml_{k = 0}^{\th-1} k^{n + 1})\)
\(= \th\)

\medskip
\(\text{Осталось }\allowbreak \text{доказать, }\allowbreak \text{что }\allowbreak \Xn\text{ -- }\text{полная }\allowbreak \text{достаточная }\allowbreak \text{статистика.}\allowbreak \)
\(\text{Тогда }\allowbreak \text{данная }\allowbreak \text{оценка }\allowbreak \text{будет }\allowbreak \text{эффективной }\allowbreak \text{в }\allowbreak \text{своём }\allowbreak \text{классе }\allowbreak \text{как }\allowbreak \text{функция }\allowbreak \text{от }\allowbreak \text{полной }\allowbreak \text{достаточной }\allowbreak \text{статистики.}\allowbreak \)

\subsubsection*{\text{ Достаточность}\allowbreak }

\(P[X_1, ..., X_n \in B | \Xn = k] = (\frac{1 }{k})^n\text{ не }\allowbreak \text{зависит }\allowbreak \text{от }\allowbreak \th\)

\subsubsection*{\text{ Полнота}\allowbreak }

\(E g(\Xn) = \suml_{k = 1}^{\th} P[\Xn = k]\cdot  g(k)\)

\(\A k \in \{1, ..., \th\}: P[\Xn = k] \ge  (\frac{1 }{\th})^n > 0\)

\(\text{Таким }\allowbreak \text{образом }\allowbreak \text{если }\allowbreak E g(\Xn) = 0,\text{ то }\allowbreak \A k \in \{1, ..., \th\}: g(k) = 0.\)
\(\text{Тогда }\allowbreak g(\Xn) = 0.\)

\let\Xn\undefined
\end{document}
